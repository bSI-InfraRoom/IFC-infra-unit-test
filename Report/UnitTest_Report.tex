\documentclass{scrartcl}

\usepackage[utf8]{inputenc}

\usepackage{amsmath}
\usepackage{babel}

\usepackage{xcolor}
\usepackage{tikz}
\usetikzlibrary{arrows.meta,calc,decorations.markings,arrows,positioning,fit,intersections,patterns,scopes,datavisualization,backgrounds}
\usetikzlibrary{shapes.geometric,decorations.pathreplacing,shadings,decorations.text}

\usepackage{hyperref}
\usepackage[footnotes,definitionLists,smartEllipses,hybrid,pipeTables=true,shiftHeadings=1,tableCaptions=true]{markdown}
\usepackage[edges]{forest}

\usepackage{cleveref}

\title{Appendix A: Unit Test Documentation}
\author{InfraRoom: IFC Infrastructure Extension Deployment project}
\date{document automatically generated on: \today}

\setkeys{Gin}{width=\linewidth}

\usepackage{tabularx}

\makeatletter
\def\markdownLaTeXReadAlignments#1{%
\advance\markdownLaTeXColumnCounter by 1\relax%
\if#1d%
\addto@hook\markdownLaTeXTableAlignment{p{0.5\textwidth}}%
\else%
\addto@hook\markdownLaTeXTableAlignment{#1}%
\fi\ifnum\markdownLaTeXColumnCounter<\markdownLaTeXColumnTotal\relax\else\expandafter\@gobble%
\fi\markdownLaTeXReadAlignments%
}
\makeatother

\begin{document}

\maketitle

\section{Summary}
\label{sec:summary}

This is a summary document of all approved unit tests submitted during the IFC Infrastructure Extensions Deployment project.

This document has been automatically produced from the \texttt{readme} files of individual unit tests.
The originating documentation together with all mentioned files can be obtained
 from the project's official GitHub repository\footnote{\url{https://github.com/bSI-InfraRoom/IFC-infra-unit-test}}.

\subsection{Schematics}
\label{sec:schematics}

The following figure represents the dependencies between the individual unit tests.
Unit tests are represented as boxes, while the arrows point from the dependant to the independent unit test.
Each unit test has a link to the corresponding section of this document in the upper right corner.

\tikzset{every label/.style={xshift=1ex, text width=10ex, align=left, 
                             inner sep=1pt, font=\footnotesize}}
\tikzset{pr/.style={dashed}}
  \begin{forest}
    for tree={              % style of tree nodes
      font=\footnotesize,
      draw, semithick, rounded corners,
            align = center,
        inner sep = 3mm,
                            % style of tree (edges, distances, direction)
             edge = {draw, semithick, latex'-},
    parent anchor = east,
     child anchor = west,
             grow = east,
    forked edge,            % for forked edge
            l sep = 9mm,   % level distance
         fork sep = 4mm,    % distance from parent to branching point
                }
    [Setup-1,name=boiler1,label=\labelcref{sec:project_setup_1}
      [Spatial-1,name=spatial1,label=\labelcref{sec:spatial_1}
        [Spatial-5,name=spatial5,label=\labelcref{sec:spatial_5}
        ],
        [Spatial-4,name=spatial4,label=\labelcref{sec:spatial_4}
          [Drainage-1,name=drainage1,label=\labelcref{sec:drainage_1}
          ],
          [Drainage-2,name=drainage2,label=\labelcref{sec:drainage_2}
          ]
        ],
        [Spatial-3,name=spatial3,label=\labelcref{sec:spatial_3}
        ],
        [Spatial-2,name=spatial2,label=\labelcref{sec:spatial_2}
        ],
      ],
      [Setup-2,name=boiler2,label=\labelcref{sec:project_setup_2}],
      [TIN-1,name=tin1,label=\labelcref{sec:tin_1}
        [GeoRef-1,name=georef1,label=\labelcref{sec:georeferencing_1}
          [GeoRef-2,name=georef2,label=\labelcref{sec:georeferencing_2},pr
          ],
          [GeoRef-3,name=georef3,label=\labelcref{sec:georeferencing_3},pr
          ]
        ] %{
          %\draw[-latex'] () to[out=south west,in=south] (boiler1);
        %}
        ,
        [GeoRef-Tin-1,name=georeftin1,label=\labelcref{sec:georef_tin_1}
          [GeoRef-Tin-2,name=georeftin2,label=\labelcref{sec:georef_tin_2}] 
        ],
      ],
      [Alignment-1,name=align1,label=\labelcref{sec:align_1}
        [Alignment-9,name=align9,label=\labelcref{sec:align_9},pr
          [Alignment-10,name=align10,label=\labelcref{sec:align_10},pr
          ]
        ],
        [Alignment-8,name=align8,label=\labelcref{sec:align_8},pr
        ],
        [Alignment-6,name=align6,label=\labelcref{sec:align_6},pr
          [Alignment-7,name=align7,label=\labelcref{sec:align_7},pr
          ]
        ],
        [Alignment-4,name=align4,label=\labelcref{sec:align_4},pr
          [Alignment-5,name=align5,label=\labelcref{sec:align_5},pr
          ],
        ],
        [Alignment-3,name=align3,label=\labelcref{sec:align_3},pr
        ],
        [Alignment-2,name=align2,label=\labelcref{sec:align_2}
        ],
      ],
    ]  
  \end{forest}
\clearpage

\subsection{How to read the document}

Each unit test is within its own section.
The included figures are \emph{floating} in this document, so they may appear at the top of the page,
 even though they are mentioned later in the text.
All figures are always flushed before the next section starts, so that there is no doubt.

\tableofcontents
\clearpage

% -------------------------------------------------------------------------------
\section{Project Setup}

\subsection{Project Setup 1}
\label{sec:project_setup_1}
\markdownInput{../ProjectSetup-1/readme.md}
\clearpage

\subsection{Project Setup 2}
\label{sec:project_setup_2}
\markdownInput{../ProjectSetup-2/readme.md}
\clearpage

% -------------------------------------------------------------------------------
\section{Georeferencing and Digital Terrain Models (DTMs)}

\subsection{TIN 1}
\label{sec:tin_1}
\markdownInput{../Tin-1/readme.md}
\clearpage

\subsection{Georeferencing 1}
\label{sec:georeferencing_1}
\markdownInput{../Georeferencing-1/readme.md}
\clearpage

\subsection{Georeferencing 2}
\label{sec:georeferencing_2}
Under review.%\markdownInput{../Georeferencing-2/readme.md}
\clearpage

\subsection{Georeferencing 3}
\label{sec:georeferencing_3}
Under review.%\markdownInput{../Georeferencing-3/readme.md}
\clearpage

\subsection{Georeferencing and DTM 1}
\label{sec:georef_tin_1}
\markdownInput{../Georeferencing-Tin-12d-1/Readme.md}
\clearpage

\subsection{Georeferencing and DTM 2}
\label{sec:georef_tin_2}
\markdownInput{../Georeferencing-Tin-12d-2/Readme.md}
\clearpage

% -------------------------------------------------------------------------------
\section{Alignment}

\subsection{Alignment 1}
\label{sec:align_1}
\markdownInput{../Alignment-12d-1/readme.md}
\clearpage

\subsection{Alignment 2}
\label{sec:align_2}
\markdownInput{../Alignment-12d-2/readme.md}
\clearpage

\subsection{Alignment 3}
\label{sec:align_3}
\markdownInput{../Alignment-12d-3/readme.md}
\clearpage

\subsection{Alignment 4}
\label{sec:align_4}
\markdownInput{../Alignment-12d-4/readme.md}
\clearpage

\subsection{Alignment 5}
\label{sec:align_5}
\markdownInput{../Alignment-12d-5/readme.md}
\clearpage

\subsection{Alignment 6}
\label{sec:align_6}
Under review.%\markdownInput{../Alignment-12d-6/readme.md}
\clearpage

\subsection{Alignment 7}
\label{sec:align_7}
Under review.%\markdownInput{../Alignment-12d-7/readme.md}
\clearpage

\subsection{Alignment 8}
\label{sec:align_8}
Under review.%\markdownInput{../Alignment-12d-8/readme.md}
\clearpage

\subsection{Alignment 9}
\label{sec:align_9}
Under review.%\markdownInput{../Alignment-12d-9/readme.md}
\clearpage

\subsection{Alignment 10}
\label{sec:align_10}
Under review.%\markdownInput{../Alignment-12d-10/readme.md}
\clearpage

\subsection{Alignment 11}
\label{sec:align_11}
Under review.%\markdownInput{../Alignment-12d-11/readme.md}
\clearpage

\subsection{Alignment 12}
\label{sec:align_12}
Under review.%\markdownInput{../Alignment-12d-12/readme.md}
\clearpage

% -------------------------------------------------------------------------------
\section{Spatial Structure}

\subsection{Spatial Structure 1}
\label{sec:spatial_1}
\markdownInput{../SpatialStructure-1/readme.md}
\clearpage

\subsection{Spatial Structure 2}
\label{sec:spatial_2}
\markdownInput{../SpatialStructure-2/readme.md}
\clearpage

\subsection{Spatial Structure 3}
\label{sec:spatial_3}
\markdownInput{../SpatialStructure-3/readme.md}
\clearpage

\subsection{Spatial Structure 4}
\label{sec:spatial_4}
\markdownInput{../SpatialStructure-4/readme.md}
\clearpage

\subsection{Spatial Structure 5}
\label{sec:spatial_5}
\markdownInput{../SpatialStructure-5/readme.md}
\clearpage

\subsection{Spatial Structure 6}
\label{sec:spatial_6}
\markdownInput{../SpatialStructure-6/readme.md}
\clearpage

% -------------------------------------------------------------------------------
\section{Earthworks}

\subsection{Earthworks 0}
\label{sec:earthworks_0}
\markdownInput{../Earthworks-12d-0/Readme.md}
\clearpage

\subsection{Earthworks 1}
\label{sec:earthworks_1}
\markdownInput{../Earthworks-12d-1/Readme.md}
\clearpage

\subsection{Earthworks 2}
\label{sec:earthworks_2}
\markdownInput{../Earthworks-12d-2/Readme.md}
\clearpage

% -------------------------------------------------------------------------------
\section{Geotechnics}

\subsection{Borehole 1}
\label{sec:borehole_1}
\markdownInput{../Borehole-1/Readme.md}
\clearpage

% -------------------------------------------------------------------------------
\section{Drainage System}

\subsection{Drainage System 1}
\label{sec:drainage_1}
\markdownInput{../DrainageSystem-1/Readme.md}
\clearpage

\subsection{Drainage System 2}
\label{sec:drainage_2}
\markdownInput{../DrainageSystem-2/Readme.md}
\clearpage




\end{document}
